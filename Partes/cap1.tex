\chapter{INTRODUCTION}\label{chap:introduction}

% What are proteins
There are four main types of macromolecules essential to life as we know: proteins,
carbohydrates, lipids and nucleic acids. Proteins are responsible for several
functions in the human body, such as structural, regulatory and metabolic,
where its functionality is directly dependent on the three-dimensional
structure conformation~\cite{kihara2014protein}.

% Why they are important
In cases where this conformation is incorrectly structured (or misfolded) due to
some environmental or genetic influence, a series of problems usually arrive.
Proteins can gain toxic properties or suffer loss of functionality,
where it can cause damage to cells,
tissues, and organs. The diseases rooted in abnormal proteins formation or somehow
caused by them are known as proteinopathies~\cite{shrestha2015yeast}.
In light of this, knowing the conformation of a protein can give insights on its
functionality, aid the design of drugs and help understand, treat and even prevent diseases.

% How do we find out the structure of a protein
The \ac{DNA} (or \ac{RNA} in some less complex forms of life) encodes the information
required to synthesize a protein. From the \ac{DNA}, the information is used to
synthesize a protein through the process of translation and transcription. The
construction of the protein happens by chaining a series of amino acids, which are
held together with peptide bonds. This chain of amino acids is called the primary
sequence, and it can uniquely identify a protein. Determining the primary
sequence is a very straightforward process and relatively cheap and can be
achieved by processes such as protein mass
spectroscopy~\cite{domon2006mass,covey1999protein}
or via Edman Degradation~\cite{edman1967protein}.

On the other hand, determining the three-dimensional structure of a protein is a
much harder and expensive process~\cite{guntert2004automated}.
Currently, the two main methods are
\acf{NMR} and X-Ray Crystallography. Both methods take more
time, money and are susceptible to have errors in the results.

Due to these differences in determining the sequence and structure of a
protein, there is a significant difference between the number of known protein
primary and tertiary structures. There are \num{158532418} protein sequences 
catalogued in UniProtKB/TrEMBL\footnote{\url{https://www.ebi.ac.uk/uniprot/TrEMBLstats}},
whereas only \num{154243} protein structures are catalogued in
the \ac{PDB}\footnote{\url{https://www.rcsb.org/}}\footnote{Numbers from June 26, 2019}.
This is often referred as the protein gap, and currently measures three orders of magnitude.
It is worth noting that this number considers repeated proteins, which might have
more than one structure associated from one or more repeated experiments. If
structures with homologous sequences are removed, this number drops down to
\num{37475}\footnote{\url{http://www.rcsb.org/pdb/results/results.do?tabtoshow=Current&qrid=B237D5AC}},
further increasing the gap to four orders of magnitude.

% What is the protein structure prediction
Given the ever so growing protein gap, several scientists from the fields of
mathematics, physics, chemistry, biology, and computer science have worked for
decades with the common goal of being to predict the conformation of a sequence
of amino acids and help close this gap. This problem is known as the
\acf{PSPP} and has been one of the leading open problems
in structural biology and computer science~\cite{dorn2014three}. The \ac{PSPP} consists of
taking an amino acid sequence as input and outputting a prediction of the protein structure.

% Why it is hard
Solving the \ac{PSPP} has proven to be a very challenging problem, since that
after more than 20 years no general viable solution has been found. From a
strictly computational point of view, the \ac{PSPP} is expected to be a
$\mathcal{NP}$-hard problem. No proof of the $\mathcal{NP}$-hardness of the
problem has been found for the more complex and accurate models, such as the
full atom model~\cite{rohl2004protein}. However, the model which is arguably
considered the less complex one, the \ac{HP} model, has been proved to be a
$\mathcal{NP}$-hard problem~\cite{berger1998protein}. Furthermore, several
other models such as the triangular 2D and 3D \ac{HP}
model~\cite{agarwala1997local} and the \ac{FCC} model~\cite{hoque2007protein},
which have an increasing level of complexity, have been found to be
$\mathcal{NP}$-hard problems.  Therefore, it is reasonable to expect the full
problem to be $\mathcal{NP}$-hard as well.

From a practical point of view, each amino acid in the protein has between 10
to 27 atoms, where each one requires three variables to describe its position.
Except for the smallest of the proteins, they will have hundreds or thousands
of variables, and can even reach millions assuming that all atoms have the
three variables. Therefore, one can consider that the \ac{PSPP} is a
\ac{LSOP}~\cite{mahdavi2015metaheuristics}, which happens to be a very
challenging problem.

% What are the main methods of solving it (ab initio, threading modeling, homology modeling)
There are two main approaches to solving the \ac{PSPP}: knowledge-based methods
and \textit{ab initio} methods. Knowledge-based methods have two subgroups:
Threading and Homology-based method. Threading modeling is based on the use of
templates that represents the vast majority of the proteins. These templates
are them combined and aligned in order to predict the protein conformation.
Homology-based methods rely on homologous proteins having the same (or similar)
conformation. Therefore, if a protein has an amino acid sequence very similar
to a protein of known structure, it will probably have a similar structure.
Homology methods require homologous protein, whereas Threading methods relies
on the templates being representative of the (unknown) target conformation. In
contrast with Knowledge-based methods, \textit{ab initio} methods do not rely
on previously known information. The main characteristic of \textit{ab initio}
method is that it operates by building the conformation from scratch guided by
an energy function. Its advantage over the other
methods is that is can (theoretically) predict any protein, regardless of a
presence of a template or homologous protein. Furthermore, it can lead to
further insights on the folding process. The \textit{ab initio} method itself
can be further divided into pure \textit{ab initio} methods, where only
physicochemical properties are used, and \textit{de novo} methods, where
statistical data gathered over the years is utilized to improve the predictions.
It is worth noting that some authors does not utilize this distinction, including
CASP itself. The present work focuses on solving the \ac{PSPP} with an
\textit{ab initio} method. More specifically, with an \textit{de novo} approach.

% How can proteins be represented computationally (on and off-lattice)
The \textit{in silico} modeling and representation of proteins are split in two
groups: On lattice and off-lattice models. The former limits the angles between
the amino acids in discrete increments, in such way that the protein is limited
to be placed over a point in the lattice and can be considered to be a discrete
problem. It also uses a much-simplified protein model, where each amino acid is
very often represented as a sphere or as a point in space~\cite{berger1998protein}.
The latter does not force the limitation of using a
lattice and can be considered as a problem with
a continuous search space. \textit{Off lattice} models have several
sub-models, with different levels of complexity and accuracy. The main models
found in the literature are $C_\alpha$ Coordinates, all heavy atoms
coordinates, backbone and centroid coordinates, backbone and side chain torsion
angles, and all atoms coordinates~\cite{dorn2014three,rohl2004protein}.
This work uses two models in its proposals: Backbone and centroids, and backbone and side chains, both using torsion angles.

% Energy potential
Solving the \ac{PSPP} in an \textit{ab initio} approach requires an energy function
which must be able to estimate the potential energy of a conformation. Since a
protein \textit{in natura} will fold towards the point of least potential
energy, by estimating its potential energy \textit{in silico}, it is possible to
use an optimizer to find the global minimum and give a prediction of the
protein structure. The computational representation of this potential energy is
an approximation of the real world interactions between the atoms and aminoacids
in the protein~\cite{alford2017rosetta}. Therefore,
different energy representations can lead to
significantly different conformations. Furthermore, more accurate energy
functions will take more processing power to be evaluated and may not be
proportionally more accurate~\cite{xu2012ab}. At this point,
there is a clear trade-off between time and accuracy.
This work uses the energy functions and protein representations provided by
Rosetta~\cite{rohl2004protein,leaver2011rosetta3}.

% What are metaheuristics and natural computing
Given that the \ac{PSPP} can be represented as a $\mathcal{NP}$-hard optimization
problem, a robust set of solvers for this kind of problem are
\ac{MH}, and in particular, bio-inspired Meta Heuristics~\cite{kar2016bio}.
Some of the most employed \ac{MH}s in the literature are the
\ac{GA}~\cite{holland1992genetic} and the
\ac{DE}~\cite{storn1997differential}. Both of them are
\ac{EA} due to their inspiration by the Darwinian process of natural
selection. As all \ac{EA}s, \ac{GA} and \ac{DE} relies on a population
(set of solutions) where an
individual (a candidate solution) compete among each other where the fittest (the one
with the best objective function) will have a higher chance of surviving into
the next generation.

% What is parameter control and why do we need it
Bio-inspired algorithms are known to have several parameters and to have their
performance dependant on both the problem instance and the set of parameters in
use~\cite{parpinelli18review}.
Finding the best set of parameters for a given \ac{MH} is a time-consuming
process since it requires the runs of the same \ac{MH} for the same instances
several times to assess its performance. Furthermore, using a different set of
instances on the \ac{MH} with the previously found set of parameters may give a
sub-optimal performance, requiring the same process to be rerun.  This process
is known as offline parameter control. One way to avoid this time-intensive
process is to enable the \ac{MH} to be able to detect parameters have the best
improvement potential and which operations lead more successful explorations of
the solution space. The process of changing the parameter while the algorithm
is running is known as online parameter control and is an active area of
research.

% What are hybrid algorithms
Furthermore, given the high complexity of the problem, a blind optimizer might
underperform due to the roughness of the energy function and the large
search space. In this case, it is possible to employ a hybrid
algorithm~\cite{blum2011hybrid}. A hybrid algorithm employs an optimizer which has no
knowledge of the problem domain with a procedure that has. This allows for the
optimization procedure to be guided to search directions which are more
relevant, and thus, possibly improve the overall performance. Given the
high dimensionality of the problem and its multimodality, this becomes a
requirement in order to improve the overall prediction quality.

% Close the intro. How are we planning on solving PSPP.
This work has as its primary goal to study and solve the \ac{PSPP}. For this an
off-lattice \textit{ab initio} method will be used. The proteins will be
represented computationally as torsion angles of the backbone and side chain
centroids. The \textit{ab initio} model will be optimized using a hybrid
optimizer based on \ac{MC} and Local Search with online parameter control.
More over, a passive control of population diversity will be employed in
conjunction with a conformational clustering routine in order to identify
promising regions of the search space.


\section{Motivation}\label{sec:chap1_motivation}

% Relevancy and Applications of the problem
The \ac{PSPP} is currently one of the leading open problems in computer science and
structural bioinformatics, and as such, it is a highly relevant one. Solving
the \ac{PSPP} has numerous implications over many fields. It can help to understand
proteinopathies better and aid the development of new smart drugs.
Also, it can give access to research groups that cannot afford the use of the
machinery required to determine the conformation of proteins and by doing so
help to generate further knowledge.

% Difficulty of the problem
Due to the $\mathcal{NP}$-hardness of the problem and its inherent complexity,
being able to solve it in a reasonable time frame can lead to better
understanding of the solving process of this kind of problems. It also can lead
to better understanding of the solving mechanisms as well, giving insight on
how to design better optimizers.

\section{Objectives}\label{sec:chap1_objectives}

This work has as its main objective to explore the \ac{PSPP} and to develop a
competitive solver for the off-lattice \textit{ab initio} \ac{PSPP} using a
hybrid optimizer with online parameter control. For this, several secondary
objectives have to be achieved as well:

\begin{itemize}
    \item Propose a flexible and configurable framework for predicting protein conformations;
    \item Assess the performance of the proposed method on several proteins;
    \item Compare the reached performance against the state-of-the-art approaches.
\end{itemize}

% This work has three hypothesis to test and/or confirm. They are:
% \begin{enumerate}
%     \item ??;
% \end{enumerate}

\section{List of Publications}\label{sec:list_of_publications}

\begin{itemize}
    \item SILVA, R. S.; PARPINELLI, R. S. A multistage simulated annealing for protein structure prediction using rosetta. Anais do Computer on the Beach, p. 850–859, 2018. Best paper award.
    \item PARPINELLI, R. S. et al. A review of technique for on-line control of parameters in swarm intelligence and evolutionary computation algorithms. International Journal of Bio-Inspired Computation (IJBIC), 2019. v. 13. p. 1-20.
    \item SILVA, R. S.; PARPINELLI, R. S. A self-adaptive differential evolution with fragment insertion  for  the  protein  structure  prediction  problem.  In:  SPRINGER. International Workshop on Hybrid Metaheuristics. 2019. p. 136–149.
\end{itemize}

\section{Document Structure} \label{sec:chap1_document_structure}

This document is organized as follows:
Chapter~\ref{chap:systematic_literature_review} presents the background
required to understand this work, the literature review and the related work;
Chapter~\ref{chap:methodology} presents the proposed methodology;
Chapter~\ref{chap:experiments_and_results} presents the experimentation,
results and its respective analysis;
Chapter~\ref{chap:final_considerations} presents the final considerations for
this work and future research directions.

