\chapter{Final Considerations and Future Work}
\label{chap:final_considerations}

% Intro to the final considerations
Proteins are macromolecules found in all known living forms and are
considered to be the working horses of the cells. The biological
function of any given protein is directly dependent on its
function, in a way that knowing it is a very valuable information
for several fields of science. The
%Human Genome\footnote{\url{https://www.genome.gov/10001772/all-about-the--human-genome-project-hgp/}}
Human Proteome\footnote{\url{https://hupo.org/human-proteome-project}} project
lead to an exponentially increasing number of sequenced proteins.
However, due to the price and time efforts required to structure
these proteins, a gap quickly formed between the number of sequenced
and the number of structured proteins. This gap now spans more than 3 orders
of magnitude. With that in mind, computer scientists, mathematicians,
physicists, chemists and biologists have been working to accurately
predict the structure of a protein from its primary
sequence and it is known as \textit{ab initio} PSPP.

In light of this, this work proposed two methods, ppf-mc and
ppf-remc, to attempt to solve the problem. The proposed methods make use of the \texttt{score0}, \texttt{score3} and \texttt{scorefxn} energy functions
from Rosetta. The protein was modeled using a full atomic model of the backbone,
manipulated using fragments and torsions angles, as well as side chain centroids.
In later stages of the prediction the model was switch to a full atomic
representation of the protein. The proposed method is capable of outputting
several conformations that are different from each other, this is accomplished
by using a clustering process. This, in turn, allows for a human expert to
evaluate and choose the best conformation based on its knowledge and intuition.

% RMSD and Energy with sade-remc and rosetta
The proposed method were compared against a method from a previous work, SADE-REMC, as
well as rosetta. In a first stage, the two proposed methods were compared
against SADE-REMC using both the RMSD and potential energy.
For both metrics a rigorous set of statistical tests was applied, which identified that
the proposed methods consistently outperformed the competitor both with lower
RMSD and lower potential energies. In a second stage, rosetta, one of the
winners and top competitors at CASP was chosen to be compared against, as a
representative of state-of-the-art. The same statistical framework was applied.
The results indicates that the proposed methods are able to give equal or better
predictions than rosetta for the majority of proteins.

% Convergence
The behavior of the two proposed methods was then inspected and analysed. It was
found that PPF-MC maintained an average pairwise RMSD consistently higher than PPF-REMC.
This lead to more variation in the clusters centroids, which in turn lead to
better predictions. PPF-REMC suffered from premature convergence and stagnation.
\textcolor{red}{
The lack of active control of convergence and populational diversity could be
considered as point worth being addressed.
}
% For one of the targets, this difference was
% of $6\AA$. On the remaining targets this difference varied between $1$ to about
% $2\AA$. One of the properties of PPF-REMC is the capacity of one trajectory to
% effectively copy and replace another, which may lead to faster convergence and
% exploitation. However, as observed, this lead to premature convergence and, in
% some cases, worst results. Meanwhile, PPF-MC was able to cope well with the
% clustering phase of the resulting conformation, thanks to its high diversity.
% Nevertheless, during this analysis, it was observed by tracking the best energy
% and average energy, that FFI was worsening the convergence for some methods.
% Due to its nature, for 1enh and 2mr9 it was observed that despite the greedy
% selection of conformation FFI was making the energy increase faster than the
% optimizer could keep up.

% MC vs REMC
With data from the convergence of the two proposed methods and statistical
analysis, a direct comparison of both methods was realized. When considering
only the RMSD there was statistical significance only for one protein, 1rop.
However, when considering the potential energy, PPF-MC had statistically
significant better energies for $5$ out of $10$ protein targets. Moreover,
despite the statistical significance of RMSD for only one target, PPF-MC had
the conformations with the best RMSD for $5$ targets. This, plus the fact that
PPF-MC has more diversity combined with the conformational clustering leads to
the conclusion that PPF-MC is the superior method. 

% Repacking Impact
% An analysis to assess the impact of repacking was also realized. The goals were
% twofold. Firstly, there was the interest in verifying that the repacking process
% was not misleading or interfering in the results. Secondly, just the plain
% documentation of how much repacking impacted the process was desirable. This
% information was not found in the literature.
Based on an analysis of repacking, it was found that
the repacking process had a small negative impact in the energies of the
proposed methods. Before repacking, both proposed methods had more results that
were better than rosetta with statistical significance. After applying repacking,
this number decreased. The mean change in RMSD was also documented. It was found
to be very low. In most cases it was lower than $0.5\AA$ on average.

% Processing time and function evaluations
% The processing time was also documented. There was no significant difference
% between the proposed methods. The main goal of analysing the processing time was
% to give a time-frame of how long it took for the prediction to be made. Along
% with this, an analysis of the total of function evaluations spent also was
% presented. The methods, on average, take less than 20 minutes to run, even for
% the bigger proteins. For the smaller proteins, the targets can be predicted in
% 10 minutes or less.

% The optimization phase of the proposed method uses 1 million of function
% evaluations, using the \texttt{score3} energy function. This is were the bulk
% of processing happen. Before this phase, the \texttt{score0} energy function
% can spend up to 10 thousand function evaluations for generating the initial
% population. Finally, in the post processing phase, the Hooke-Jeeves search
% procedure spends on average 20 thousand function evaluations. A value relatively
% small compared to the budged from the optimization phase. Nevertheless, this
% data is made available in order to give a solid ground for future comparisons
% with the presented methods.

% Literature comparison
To be able to compare the proposed methods against the literature, a literature
review was realized, which selected the 10 most common proteins.
Based on this, it was found that the proposed methods are
competitive with several methods in the literature. For two targets 1wqc and
2mr9 the proposed methods were able to overpass the best result found in the
literature. For 1rop and 1crn, the two proteins used more often in the
literature, one of the proposed methods were able to get the second best
prediction. The same occurred for 1enh. For the remaining targets the proposed
methods achieved results that were often close to the best results available.

% GDT-TS and TMScore
A second analysis of prediction error was realized using GDT-TS and TM-Score,
which provided further insights, complementary to RMSD. Firstly, it was observed
that GDT-TS appears to be more strict about the results than RMSD. Furthermore,
TM-Score appears to be even stricter. Another key-point observed, is that the
three metrics does not always agree, where they gave values significantly
apart. Nevertheless, studying the GDT-TS and TM-Score gave the insights that
the prediction could be improved even further for several of the proteins which
judging just by RMSD alone were considered near native conformations. In the end,
the raw data from these observations was made available in order to allow for
future works to compare against ours using rigorous statistical methods.

% Visual Analysis
A visual analysis was utilized to try and identify what the main prediction
errors were. Four proteins were detected as having major errors. From these,
three had its errors tracked down to the predicted secondary structure, which is
and input of the proposed methods. As such, the predictions were made using
incorrect data, leading to a sub par performance. The dependency on
uncertain data is potentially one of the major weak-spots of the proposed methods.
Nevertheless, for the majority of the predictions correct predictions were made.

\section{Future Works}\label{sec:future_works}

% Secondary Structure Prediction
A future research direction
would be to employ more than one predictor, in a way that if one is wrong,
there is a chance that the other is not. It is also possible to research in the
direction of making the Tertiary Structure Prediction either less dependent on
the Secondary Structure Prediction or making it more resilient to faulty
information.

% Better frags
The use of better generated fragments is also a research direction, since a big
portion of the conformation sampling method is dependent on it. A fragment
library of poor quality will lead to a poor prediction.
% More hybridization
The further hybridization with methods can also be considered. One possible
direction is the inclusion on gradient descent methods to quickly find nearby
areas of low energy.
% Diversity
%Another weak point of this work is that there is no passive or active method of
%controlling the diversity of the population.
As show in works~\citeonline{narloch2016diversification,simoncini2017balancing},
actively or passively acting on populational diversity can lead to
better predictions. As such, managing populations diversity might further
extend the potential of using conformational clustering.

% Better conformation sampling
A recent trend worth exploring is the focusing the search on loops and coil
regions of the proteins. These areas usually have a lower acceptance rate of
perturbation due to the high impact they have on the overall conformation.  In
contrast, $\alpha$-helices are relatively stable and have a much higher
acceptance rate while having a low impact in the overall structure.

% Add restarts
Running a method for too long will most likely given diminishing returns, as
the diversity decreases. One way to give better predictions is to periodically
reset parts of the population or the full population if stagnation occurs. This
can allow for the method to run longer and possibly lead to better predictions.
In particular, partial population re-initialization can allows for old information
to contribute to the newer generated individuals.

% LS for SaDE
The proposed methods applied Hooke-Jeeves pattern search as one of the last steps.
However, on direction would be to use Hooke-Jeeves, or other local search
procedure as an operator for SaDE. This would allow for the method to quickly
find local regions of good potential energy. However, this could also lead to
premature convergence, and as such, it must have a counter acting way of
preventing the population from getting stuck. Further, due to the nature of
SaDE, it is possible that the local search operators perform too well at the
start, making SaDE heavily prefer it. Then, at latter stages, they will
have diminishing returns. However, SaDE will keep heavily applying the
same operators over and over again. Therefore, it might be required for another
strategy to be utilized to control which operator is applied.
