\imprimircapa

\imprimirfolhaderosto*

% ---
% Caso a Biblioteca da UDESC forneça, utilize o comando
% ---
% \begin{fichacatalografica}
%     \includepdf{fig_ficha_catalografica.pdf}
% \end{fichacatalografica}

% ---
% Geração da Ficha Catalográfica Via LaTeX
% ---
\begin{fichacatalografica}
  \vspace*{\fill}					% Posição vertical
  \begin{center}					% Minipage Centralizado
    \begin{minipage}[c]{12.5cm}		% Largura

      \imprimirautor

      \hspace{0.5cm} \imprimirtitulo  / \imprimirautor. --
      \imprimirlocal, \imprimirdata-

      \hspace{0.5cm} \pageref{LastPage} p. : il. (algumas color.) ; 30 cm.\\

      \hspace{0.5cm} \imprimirorientadorRotulo~\imprimirorientador\\

      \hspace{0.5cm}
      \parbox[t]{\textwidth}{\imprimirtipotrabalho~--~\imprimirinstituicao,
      \imprimirdata.}\\

      \hspace{0.5cm}
      1. Tópico 01.
      2. Tópico 02.
      I. Prof. Dr. xxxxx.
      II. Universidade do Estado de Santa Catarina.
      III. Centro de Ciências Tecnológicas.
      IV. identificação xxxx\\ 

      \hspace{8.75cm} CDU 02:121:005.7\\

    \end{minipage}
  \end{center}
\end{fichacatalografica}

% ---
% Folha de Aprovação
% ---
% Exemplo de folha de aprovação antes da Banca. Após isso, incluia o pdf digitalizado com as assinaturas%
% \includepdf{folhadeaprovacao_final.pdf}
\begin{folhadeaprovacao}

  \begin{center}
    {\ABNTEXchapterfont\bfseries\imprimirautor}
    \vspace{6em}

      \ABNTEXchapterfont\bfseries\imprimirtitulo

  \end{center}
    \vspace{1em}
    {\justify
    This content was submitted to examination board as partial fulfillment to obtain the title of
      {\ABNTEXchapterfont\bfseries Master in Applied Computing},
      area of concentration in "Computational Systems", by the Graduate Program in Applied Computing at the College of Technological Science of Santa Catarina State University.}

      % 		{\justify
      % 		This Master thesis was considered adequate to obtain the title of
      %     	{\ABNTEXchapterfont\bfseries Master in Applied Computing},
      %    		area of concentration in "Computational Systems",
      %    		and approved in its final form by the Graduate Program in Applied Computing at the College of Technological Science of Santa Catarina State University.}

  \vspace{3em}
  \noindent
  {\bfseries Examination Board:}
  \assinatura{\textbf{\imprimirorientador} \\ Advisor}
    \assinatura{\textbf{Dr. Rafael Rodrigues Obelheiro} \\ CCT-UDESC}
    \assinatura{\textbf{Dr. Adriano Fiorese} \\ CCT-UDESC}

    \vspace*{\fill}
    \begin{center}
      \imprimirlocal,\,\imprimirfulldata
    \end{center}
\end{folhadeaprovacao}

% ---
% Dedicatória
% ---
\begin{dedicatoria}
  To my Father that is no longer with us and to my Mother that always stood by me.
\end{dedicatoria}

% ---
% Agradecimentos
% ---
% TODO: Update agradecimentos
\begin{agradecimentos}
  Firstly I would like to thank my mother, Tatiana. Without her continuous and never ending support
  I would not been here. Not only did she financially and emotionally support me, she also
  did encourage me since I was very young to study and go after my dreams. This shaped me
  into who I am.

  I also would like to deeply thank Rafael Parpinelli for supporting me over this journey
  and helping me push the limits of my knowledge. I also thank all my friends that were with
  me since my first day at the university. In special Andressa Umetsu, Rafael Castro, Vinicius Zuchi,
  Tiago Heinrich,
  Aurelio Grott, Mateus Boiani, William Pereira, Karll Henning, Dalmo Neto, Lucas Zanatta and many others.
  They all made my days more bearable and fun.
  To all the teachers that served both as friends and a role models,
  in special professor Omir. I had the pleasure of working for more than two years with him.
  The experience that I gained while working with him helped define who I am today as an academic.
  To my team lead at JobScore, Thiago, who can understand what is like to be working and doing a master
  degree at the same time and helped along the path as well.

  Finally I would like to thank Andressa Umetsu for becoming one of the closest friends that
  I have. Her friendship made me a happier person.
\end{agradecimentos}

% ---
% Epígrafe
% ---
% TODO: Maybe get a new quote?
\begin{epigrafe}
  ``Ever tried. Ever failed. No matter. Try Again. Fail again. Fail better.''
  \\
  \par
  Samuel Beckett
\end{epigrafe}

% ---
% RESUMOS
% ---

% ---
% Ao usar o modo twoside (anverso e verso) o resumo não se posiciona na página ímpar.
% Dessa forma, deve-se forçar o resumo para iniciar em uma página ímpar, usando o seguinte comando:
% ---

%\newpage\null\thispagestyle{empty}\newpage

% Inglês
% TODO: Update resumo en
\begin{resumo}
  Lorem ipsum dolor sit amet, consectetur adipiscing elit. Donec commodo risus mollis turpis posuere, quis iaculis odio suscipit. Vestibulum vehicula augue id lacus aliquam feugiat. Orci varius natoque penatibus et magnis dis parturient montes, nascetur ridiculus mus. Duis mollis purus sed sollicitudin fermentum. Etiam porta in tortor eget pretium. Nunc ac justo egestas nibh laoreet euismod. Donec a erat non lorem porttitor facilisis ut a tortor. Curabitur in finibus eros. Etiam scelerisque, mauris vel aliquam fringilla, nunc nisl egestas metus, sed facilisis urna risus et quam. Duis vitae dapibus dolor.

  Aenean ultricies metus a lacinia mollis. Sed commodo tempor ipsum sed vehicula. Class aptent taciti sociosqu ad litora torquent per conubia nostra, per inceptos himenaeos. Nunc ut vulputate eros. Maecenas aliquet, dui ut cursus sollicitudin, lorem lacus accumsan urna, at hendrerit elit purus in nunc. In dolor quam, auctor nec maximus a, lacinia id metus. Vestibulum ante ipsum primis in faucibus orci luctus et ultrices posuere cubilia Curae; Pellentesque euismod eleifend sapien, eu tincidunt turpis porta non. Quisque finibus bibendum pellentesque.
  \\
  \vspace{\onelineskip}
  \noindent
  \textbf{Keywords}: Differential Evolution, Hybrid Methods, Monte Carlo, Fragment Insertion, Protein Structure Prediction Problem, Parameter Control.
\end{resumo}

% ---
% Ao usar o modo twoside (anverso e verso) o resumo não se posiciona na página ímpar.
% Dessa forma, deve-se forçar o resumo para iniciar em uma página ímpar, usando o seguinte comando:
% ---

%\newpage\null\thispagestyle{empty}\newpage

% Português
% TODO: Update resumo pt-br
\begin{resumo}[Resumo]
  \begin{otherlanguage*}{brazil}
    Lorem ipsum dolor sit amet, consectetur adipiscing elit. Mauris ex ante, cursus aliquam interdum vitae, varius sit amet leo. Etiam eleifend aliquet justo, in commodo risus. Nam quis libero et odio suscipit egestas. Etiam porta, sem vel venenatis pulvinar, dolor nisl vestibulum est, et suscipit nisl libero et erat. Mauris ullamcorper aliquam eros, id pellentesque mi sagittis vel. Maecenas id neque volutpat, varius purus ac, cursus lacus. Maecenas vel tincidunt mauris, eu tempus dui. Vivamus consequat, urna at lobortis tincidunt, purus nisl ultricies lacus, nec efficitur elit ipsum nec nisl. Nam tincidunt commodo velit, sed hendrerit velit condimentum nec. Curabitur tempor molestie scelerisque. Proin dignissim sollicitudin vulputate. Aliquam nec lorem eu dolor tincidunt scelerisque. Pellentesque nisl risus, dapibus vel tortor nec, ultrices venenatis lectus. Nam mattis, tortor vitae semper fermentum, ligula felis facilisis eros, ac commodo nulla lorem quis mauris. Nam blandit eros eget neque facilisis, ut suscipit est volutpat. Cras eu faucibus ante.

    Nunc quis quam dui. Suspendisse at posuere nibh. Maecenas ligula arcu, ullamcorper eu massa in, porta accumsan sapien. Etiam cursus porttitor venenatis. Nullam eleifend id metus ut fermentum. Nulla viverra tempor cursus. Maecenas elit sapien, hendrerit at ornare id, aliquet pulvinar dui. In iaculis justo at tempor mattis. Nulla tempor eu nunc non tincidunt. Pellentesque neque augue, pulvinar id convallis a, consequat vitae nunc. Integer vel erat felis. Vivamus sodales enim eu ipsum congue tempus.
    \\
    \vspace{\onelineskip}
    \noindent
    \textbf{Palavras-chave}: Differential Evolution, Métodos Híbrids, Monte Carlo, Inserção de fragmento, Problema de Predição de Estrutura de Proteínas, Controle de Parâmetros.
  \end{otherlanguage*}
\end{resumo}

% ---
% Lista de Figuras
% ---
\pdfbookmark[0]{\listfigurename}{lof}
\listoffigures*
\cleardoublepage
% ---

% ---
% Lista de Tabelas
% ---
\pdfbookmark[0]{\listtablename}{lot}
\listoftables*
\cleardoublepage
