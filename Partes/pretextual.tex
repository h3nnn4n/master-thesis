\imprimircapa

\imprimirfolhaderosto*

% ---
% Caso a Biblioteca da UDESC forneça, utilize o comando
% ---
% \begin{fichacatalografica}
%     \includepdf{fig_ficha_catalografica.pdf}
% \end{fichacatalografica}

% ---
% Geração da Ficha Catalográfica Via LaTeX
% ---
\begin{fichacatalografica}
	\vspace*{\fill}					% Posição vertical
	\begin{center}					% Minipage Centralizado
	\begin{minipage}[c]{12.5cm}		% Largura
	
	\imprimirautor
	
	\hspace{0.5cm} \imprimirtitulo  / \imprimirautor. --
	\imprimirlocal, \imprimirdata-
	
	\hspace{0.5cm} \pageref{LastPage} p. : il. (algumas color.) ; 30 cm.\\
	
	\hspace{0.5cm} \imprimirorientadorRotulo~\imprimirorientador\\
	
	\hspace{0.5cm}
	\parbox[t]{\textwidth}{\imprimirtipotrabalho~--~\imprimirinstituicao,
	\imprimirdata.}\\
	
	\hspace{0.5cm}
		1. Tópico 01.
		2. Tópico 02.
		I. Prof. Dr. xxxxx.
		II. Universidade do Estado de Santa Catarina.
		III. Centro de Ciências Tecnológicas.
		IV. identificação xxxx\\ 			
	
	\hspace{8.75cm} CDU 02:121:005.7\\
	
	\end{minipage}
	\end{center}
\end{fichacatalografica}

% ---
% Folha de Aprovação
% ---
% Exemplo de folha de aprovação antes da Banca. Após isso, incluia o pdf digitalizado com as assinaturas%
% \includepdf{folhadeaprovacao_final.pdf}
\begin{folhadeaprovacao}

	\begin{center}
		{\ABNTEXchapterfont\bfseries\imprimirautor}
		\vspace{6em}

			\ABNTEXchapterfont\bfseries\imprimirtitulo
		
	\end{center}
		\vspace{1em}
		{\justify
		This content was submitted to examination board as partial fulfillment to obtain the title of
    	{\ABNTEXchapterfont\bfseries Master in Applied Computing},   
   		area of concentration in "Computational Systems", by the Graduate Program in Applied Computing at the College of Technological Science of Santa Catarina State University.}

% 		{\justify
% 		This Master thesis was considered adequate to obtain the title of
%     	{\ABNTEXchapterfont\bfseries Master in Applied Computing},   
%    		area of concentration in "Computational Systems",
%    		and approved in its final form by the Graduate Program in Applied Computing at the College of Technological Science of Santa Catarina State University.}

	\vspace{3em} 
	\noindent
	{\bfseries Examination Board:}
	\assinatura{\textbf{\imprimirorientador} \\ Advisor} 
    \assinatura{\textbf{Dr. Rafael Rodrigues Obelheiro} \\ CCT-UDESC}
    \assinatura{\textbf{Dr. Adriano Fiorese} \\ CCT-UDESC}

    \vspace*{\fill}
    \begin{center}
    	\imprimirlocal,\,\imprimirfulldata
    \end{center}
\end{folhadeaprovacao}

% ---
% Dedicatória
% ---
\begin{dedicatoria}				
To my Father that is no longer with us and to my Mother that always stood by me.
\end{dedicatoria}

% ---
% Agradecimentos
% ---
\begin{agradecimentos}
Firstly I would like to thank my mother, Tatiana. Without her continuous and never ending support
I would not been here. Not only did she financially and emotionally support me, she also
did encourage me since I was very young to study and go after my dreams. This shaped me
into who I am.

I also would like to deeply thank Rafael Parpinelli for supporting me over this journey
and helping me push the limits of my knowledge. I also thank all my friends that were with
me since my first day at the university. In special Andressa Umetsu, Rafael Castro, Vinicius Zuchi,
Tiago Heinrich,
Aurelio Grott, Mateus Boiani, William Pereira, Karll Henning, Dalmo Neto, Lucas Zanatta and many others.
They all made my days more bearable and fun.
To all the teachers that served both as friends and a role models,
in special professor Omir. I had the pleasure of working for more than two years with him.
The experience that I gained while working with him helped define who I am today as an academic.
To my team lead at JobScore, Thiago, who can understand what is like to be working and doing a master
degree at the same time and helped along the path as well. 

Finally I would like to thank Andressa Umetsu for becoming one of the closest friends that
I have. Her friendship made me a happier person. 

\end{agradecimentos}

% ---
% Epígrafe
% ---
\begin{epigrafe}	
``Ever tried. Ever failed. No matter. Try Again. Fail again. Fail better.''
\\
\par
Samuel Beckett
\end{epigrafe}

% ---
% RESUMOS
% ---

% ---
% Ao usar o modo twoside (anverso e verso) o resumo não se posiciona na página ímpar.
% Dessa forma, deve-se forçar o resumo para iniciar em uma página ímpar, usando o seguinte comando:
% ---

%\newpage\null\thispagestyle{empty}\newpage

% Português
\begin{resumo}
The Protein Structure Prediction Problem (PSPP) is currently one of the most important and more challenging open problems
in both computer science and in structural bioinformatics. Being able to accurately predict protein conformations
would have a significant impact on several fields. Proteinopathies would be better understood, intelligent protein
based drugs could be designed more easily and the overall understanding of the protein functions in organisms would
be further increased. Albeit the problem being several decades old, it still largely unsolved, being only able to
predict small proteins. As such, this work has as its main goal to attempt to improve the prediction power of
\textit{ab initio} methods by hybridizing an Differential Evolution algorithm with online parameter control with
a Monte Carlo based fragment insertion. With this, a powerful meta heuristic is utilized as the core of the
conformation sampling process with fragment insertion feeding domain specific information into the process. The
online parameter control allows the method to adapt to proteins of different characteristics and also to adapt
to different stages of the optimization process. The main contributions of this work are the study of online parameter
control in the PSPP and the impact of using hybridization. In order to access the performance of the proposed methods
they were compared against each other using statistical methods. The best performing methods were then compared against
compatible methods in the literature. Accordingly to the experimentation ran and its respective results, it was
verified that the use of online parameter control had a positive impact on the proposed methods. Furthermore, the
combined use of online parameter control and hybridization led to better results than competing methods found
in the literature.
 \\ 
 \vspace{\onelineskip}
 \noindent
 \textbf{Keywords}: Differential Evolution, Hybrid Methods, Monte Carlo, Fragment Insertion, Protein Structure Prediction Problem, Parameter Control.
\end{resumo}

% ---
% Ao usar o modo twoside (anverso e verso) o resumo não se posiciona na página ímpar.
% Dessa forma, deve-se forçar o resumo para iniciar em uma página ímpar, usando o seguinte comando:
% ---

%\newpage\null\thispagestyle{empty}\newpage

% Inglês
\begin{resumo}[Resumo]
 \begin{otherlanguage*}{brazil}
O Problema de Predição de Estrutura de Proteínas (PPEP) é atualmente um dos mais importantes e difíceis problemas 
em aberto na Ciência da Computação e na Bioinformática Estrutural.
A capacidade de poder predizer com precisão a conformação de proteínas
teria um impacto significativo em diversas áreas de pesquisa. Tais como, 
Proteinopatias seriam melhor entendidas; Drogas inteligentes baseadas em proteínas seriam mais facilmente
desenvolvidas; O entendimento geral da função das proteínas nos organismos seria melhorado.
Apesar do problema estar em aberto por diversas décadas, de modo geral ainda não foi resolvido. Atualmente
apenas pequenas proteínas podem ser preditas com precisão. À vista disto, este trabalho tem como seu objetivo
principal melhorar o poder preditivo de métodos \textit{ab initio} atráves a hibridização da Evolução Diferencial
com controle online de parâmetros acoplado com inserção de fragmentos baseada em Monte Carlo.
Deste modo, uma poderosa meta heurística é utilizada como núcleo de busca de conformações e tem a inserção
de fragmentos como um método de agregar informações sobre o domínio do problema no processo.
O uso de controle online de parâmetros permite que o método se adapte a proteínas de diferentes características
e a diferentes fases do processo de otimização. As principais contribuiçoes deste trabalho estão no estudo
do uso de controle online de parâmetros no contexto do PPEP e no impacto do uso de métodos híbridos.
Com o objetivo de avaliar o desempenho dos métodos propostos os mesmos foram comparados entre si utilizando
métodos estatísticos. Os melhores métodos foram então comparados contra métodos compatíveis na literatura.
De acordo com a experimentação conduzida e os respectivos resultado, o uso de controle online de parâmetros
teve um impacto positivo no poder preditivo do método. A combinação de controle online de parâmetros e hibridização
promoveram resultados melhores que os métodos compatíveis encontrados na literatura.
  \\
   \vspace{\onelineskip}
   \noindent 
   \textbf{Palavras-chave}: Differential Evolution, Métodos Híbrids, Monte Carlo, Inserção de fragmento, Problema de Predição de Estrutura de Proteínas, Controle de Parâmetros.
 \end{otherlanguage*}
\end{resumo}

% ---
% Lista de Figuras
% ---
\pdfbookmark[0]{\listfigurename}{lof}
\listoffigures*
\cleardoublepage
% ---

% ---
% Lista de Tabelas
% ---
\pdfbookmark[0]{\listtablename}{lot}
\listoftables*
\cleardoublepage
