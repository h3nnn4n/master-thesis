\chapter{Experiments, Results and Analysis}\label{chap:experiments_and_results}

%Present the chapter
In this Chapter the experiment design is presented. Results are shown and an in
depth analysis is conducted, including: Energy and distance metrics for
measuring the performance of the proposed methods; A rigorous statistic testing
of the proposed methods; A comparison to competing methods in the literature in
order to validate the proposed methods; Finally, a visual inspection of the
best predictions and a comparison against the native conformation.

\section{Design of Experiments}\label{sec:design_of_experiments}

%Present the hardware setup
The experiments were all conducted on a single machine using the same hardware
throughout the full experimentation length. Table~\ref{tab:machine-setup}
presents the machine utilized to run all the experiments. Each run of a
prediction method consists of a serial program that run continually without
interruption.  The experiments were run in parallel, limited to at most one
running test per core\footnote{Only physical cores were considered. No virtual
(Hyperthreading) core was involved in the computations.}.  To ensure maximum
repeatability the machine had no graphical interface enabled during the course
of the experimentation.

\begin{table}[th]
    \centering
    \begin{tabular}{r|l} \hline \hline
        Name & Value \\ \hline \hline
        Operating System & Arch Linux \\ \hline
        Kernel &  Arch Linux Kernel 4.18.16 \\ \hline
        CPU & Intel(R) Core(TM) i5-3570K CPU @ 4.20GHz \\ \hline
        Number of Cores & 4 Physical cores, no hyperthreading cores \\ \hline
        RAM & 16 GB @ 1400 MHz \\ \hline \hline
    \end{tabular}
    \caption{The Machine Setup}
    \label{tab:machine-setup}
\end{table}

The experimentation is separated into two stages. In the first stage the four
proposed methods are compared against each other and against a standard
\ac{SaDE}. The metrics utilized are the \textit{scorefxn} energy value of the
best solution and the \ac{RMSD} associated with the same conformation.  The
result was collected over 60 independent runs of each method for each target
protein. A graphical analysis is conducted in order to identify visually the
relative performance between the proposed methods.  Considering that a visual
analysis is not enough (in this case), a more rigorous numerical statistic set
of test is conducted.  The Shapiro-Wilk~\cite{wilk1968joint} normality test is
employed with a confidence level of 5\%, i.e. $\alpha = 0.05$, to assess the
presence (or lack) of a underlying normal distribution. Based on its result, a
parametric/non-parametric test is employed with a confidence level of $\alpha =
0.05$. Due to the multiple comparisons involved \v{S}idák's $\alpha$ correction
will be utilized~\cite{vsidak1967rectangular}.  The winner (or winners)
method(s) will be used in further comparison against the literature.  The
processing time for this stage is also analyzed. The time required from the
start of the initial population generation until the final full atom model is
output is measured in seconds.

% Present the protein set
In the second stage of experimentation, 10 independent runs apart from the
first stage are used to compare with related works found in the literature. A
set of proteins was selected based on previous experiments by other authors. To
be able to fully compare the proposed methods with others in the literature
only experiments using the same metrics and protein were utilized. This
decision limits the number of previous works that can be compared to this one.
Furthermore, it also limits the number of times that the proposed methods can
be ran. Nevertheless, it provides a solid comparison standard. The metrics
utilized are the best scorefxn energy function and the best \ac{RMSD} from all
runs, and the mean and standard deviation of the scorefxn energy function of
all runs. A direct comparison of the minimum energy and \ac{RMSD} will be used
as criterion for comparing the methods. % as well as an unpaired Student T
Test.
%

With this constraints and considerations a set of four proteins was assembled:
1ZDD, 1CRN, 1ENH and 1AIL. Table~\ref{tab:protein-targets} presents some
characteristics of the set of proteins chosen as test suit.  The column
\textbf{Name} contains the protein identification code as in PDB.  The
\textbf{Size} column shows the number of amino acids in the protein.  The
\textbf{Backbone Angles} column shows the number of angles in the backbone,
this also has a one to one relation to the number of variables to be optimized
for a given protein. The \textbf{Structure} column holds the secondary
structures present in the protein set represented by $\alpha$-helices or
$\beta$-sheets.

\begin{table}[bh]
  \centering
  \begin{tabular}{ l | c | c | c | c }
    \hline \hline
    Name & Size & Backbone Angles & Structure         \\ \hline \hline
    1ZDD & 35   & 105             & $2\alpha$         \\ \hline
    1CRN & 46   & 138             & $2\alpha, 2\beta$ \\ \hline
    1ENH & 54   & 162             & $3\alpha$         \\ \hline
    1AIL & 72   & 216             & $3\alpha$         \\ \hline
    \hline
  \end{tabular}
  \caption{Target proteins and their features}
  \label{tab:protein-targets}
\end{table}

% present each protein (or maybe not)

% present the two algorithms (MC vs REMC

% Present the parameters

\begin{table}[ht]
    \centering
    \begin{tabular}{r|l} \hline \hline
        Parameter & Value \\ \hline \hline
        SaDE learning Phase & 50 \\ \hline
        Population Size & 100 \\ \hline
        Function Evaluation Budget & 500000 \\ \hline
        MC/REMC Function Evaluation Budget & 100 \\ \hline \hline
    \end{tabular}
    \caption{Parameters utilized in the proposed methods}
    \label{tab:parameters}
\end{table}

\section{Energy and \ac{RMSD} Analysis}\label{sec:methods-analysis}

% Present the metrics to be utilized

% Show the energy and distance table

%Analyze energy

% Analyze RMSD

% Analyze the avg energy

% Analyze the overall results and conclude

\section{Convergence and Diversity Analysis}

\section{Parameter Analysis and Operator Usage}

\section{Forced Fragment Insertion Analysis}

\section{Repacking Impact}

\section{Processing time}

\section{Comparison with Competing Methods}

\section{GDT-TS and TM-Score metrics}

\section{Visual Representation of the Predictions}

